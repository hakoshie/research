\documentclass[a4paper,12pt,uplatex]{jsarticle}
\title{\vspace{-3cm}\Large 修士論文の要旨\\
\LARGE 医療用医薬品の市販薬への転用\\(スイッチOTC化)の影響}
\author{EM225025 橋之口浩平}
\date{}
\usepackage{amsmath}
\usepackage{amssymb}
\usepackage{amsthm}
\usepackage{ascmac}
\usepackage{bm}
\usepackage{comment}
\usepackage{color}
\usepackage{enumerate}
%\usepackage{eclbkbox}
\usepackage{empheq}
\usepackage{float}
\usepackage[dvipdfmx]{graphicx}
\usepackage[dvipdfmx]{hyperref}
%\usepackage{breakurl} %after hyperref
\usepackage{mathrsfs}
\usepackage{natbib}
\usepackage{pxjahyper}
%\usepackage{physics}
%\usepackage{sgamex}
%\usepackage{listings,jvlisting}
%\usepackage{tikz}
%\renewcommand{\labelitemi}{$\blacktriangleright$}
\PassOptionsToPackage{hyphens}{url}
\def\UrlBreaks{\do\/\do-}
\bibpunct[:]{(}{)}{,}{a}{,}{,}
\let\Iff\Leftrightarrow
\let\trm\textrm
\let\tbf\textbf
\let\tc\textcolor{}{}
\let\mf\mathfrak
\let\msc\mathscr
\let\msr\mathscr
\let\ms\mathscr
\let\ov\overline
\let\To\Rightarrow
\let\l\left
\let\r\right
\theoremstyle{definition}
\newtheorem{definition}{定義}[section]
\newtheorem{thm}{定理}[section] 
\newtheorem{prf}{証明}[section]
\newtheorem{exmp}{例}[section]
\renewcommand*{\thefootnote}{\arabic{footnote}}
\begin{document}
\maketitle{}
本研究では、日本における医療用医薬品の市販薬(以下、OTC\footnote{over-the-counterの略}とも呼ぶ)への転用(スイッチOTC化)の影響を分析する。セルフメディケーションの推進のために、医療用医薬品として使われている成分が市販薬として販売可能となることがあり、この市販薬はスイッチOTC\footnote{スイッチOTCはOTCの部分集合である。スイッチOTCでないOTCを非スイッチOTCと呼ぶことにする。}と呼ばれる。日本OTC医薬品協会は医薬品のライフサイクルとしてスイッチOTCを後発医薬品の次の段階に位置付けている\footnote{スイッチOTCの期待\url{https://www.jsmi.jp/special/switch/about/expect.html}}。
OTCによる潜在的医療費削減効果は\cite{igarashi2021}で試算してあり、レセプトの情報から置き換え可能とされているものが3200億円となっている。しかし、ここでは医療用医薬品とOTCの代替関係が考慮されておらず、実際にそのような置き換えが可能なのかについては疑問が残る。そこで本研究では、スイッチOTC化の影響を定量化するために、固定効果モデルによるイベントスタディや\cite{Berry1994}や\cite{BLP}の消費者の離散選択モデルによる需要推定を行った。データは、薬事工業生産動態統計やNDBオープンデータを主に使用した。そのほかにKEGG DRUGデータベース、スイッチOTCの発売日のデータ、薬価基準、医薬品HOTコードマスター、後発医薬品の使用割合も使用した。消費者の直面する価格に関するデータは得られなかったため、価格比の情報をもとに薬価の定数倍とした。
スイッチした成分の属する薬効分類に対する、スイッチOTCの販売開始をトリートメントとしたイベントスタディでは、OTCの出荷額については35-65\%程度上昇する影響が見られたが、医療用医薬品の出荷額や処方量にはあまり影響がないことがわかった。
需要推定ではNested Logitモデルを使用した。価格弾力性の計算からは医療用医薬品とOTCの代替関係は医療用医薬品間の代替関係よりは弱いことが示唆された。仮想的な場合と現実の消費者余剰の比較からは、スイッチOTCによる厚生の改善は年間1人あたり1200円程度、非スイッチOTCは年間1人あたり2800円程度と試算された。これらの結果は、スイッチOTCは消費者の選択肢を増やしOTCの市場を拡大させるが、医療用医薬品を代替することはあまりないことを示唆する。

関連する研究としては、以下のものがある。\cite{Stomberg2013}では、アメリカでのスイッチOTC化の影響を薬効レベルや成分レベルで中断時系列分析を用いて分析し、スイッチOTC化により医薬品の使用が増えることを示している。\cite{Keeler2002}では、禁煙補助薬のOTC化による社会的便益を操作変数法などを用いた需要関数の推定を通して計算している。\cite{Mahecha2006}はスイッチOTC化のケーススタディを行い、成功には参入のタイミングやマーケティングが大切であるとしている。\cite{medicare1995}は健康保険のカバーする範囲による医療用医薬品とOTCの使用の変化を分析し、健康保険に入っている人ほど、医療用医薬品の使用が増え、OTCの使用が減ることを確かめている。\cite{medicare2003}は2腕意思決定モデルで第二世代抗ヒスタミン薬のOTC化の影響を分析し、医療費を減少させるだけでなく、事故なども減少させ社会的にもよいものだったとしている。\cite{Iizuka2007}では日本の医療用医薬品市場を離散選択モデルで分析し、医師と患者の間のプリンシパルエージェント問題を指摘している。

先行研究との違いは、日本のデータを使っている点とスイッチOTC化の影響を離散選択モデルで分析している点である。\cite{JLE2002}では特に医療用医薬品市場ではプリンシパルエージェント問題が存在することから離散選択モデルによる需要推定は難しいとされている。しかし、日本では処方箋受取率(医薬分業率)が2020年に75\%を超える\footnote{医薬分業とは\url{https://www.nichiyaku.or.jp/activities/division/about.html}}など医薬分業が進んできていることや患者が後発医薬品を薬局で希望できるようになってきていることからその問題は和らいでいると考えられる。
\bibliography{../thesis} %hoge.bibから拡張子を外した名前
\bibliographystyle{jecon}
\end{document}