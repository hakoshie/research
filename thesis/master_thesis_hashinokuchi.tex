\documentclass[a4paper,11pt,uplatex]{jsarticle}
\title{\vspace{-3cm}医療用医薬品の市販薬への転用\\(スイッチOTC化)の影響}
\author{橋之口浩平\thanks{一橋大学大学院経済学研究科修士課程2年 e-mail: em225025@g.hit-u.ac.jp}}
\date{}
\usepackage{amsmath}
\usepackage{amssymb}
\usepackage{amsthm}
\usepackage{ascmac}
\usepackage{bm}
\usepackage{comment}
\usepackage{color}
\usepackage{enumerate}
%\usepackage{eclbkbox}
\usepackage{empheq}
\usepackage{float}
\usepackage[dvipdfmx]{graphicx}
\usepackage[dvipdfmx]{hyperref}
\usepackage{mathrsfs}
\usepackage{natbib}
\usepackage{pxjahyper}
\usepackage{bbm}
%\usepackage{physics}
%\usepackage{sgamex}
%\usepackage{listings,jvlisting}
%\usepackage{tikz}
%\renewcommand{\labelitemi}{$\blacktriangleright$}
\PassOptionsToPackage{hyphens}{url}
\def\UrlBreaks{\do\/\do-}
\bibpunct[:]{(}{)}{,}{a}{,}{,}
\let\Iff\Leftrightarrow
\let\trm\textrm
\let\tbf\textbf
\let\tc\textcolor{}{}
\let\mf\mathfrak
\let\msc\mathscr
\let\msr\mathscr
\let\ms\mathscr
\let\ov\overline
\let\To\Rightarrow
\let\l\left
\let\r\right
\theoremstyle{definition}
\newtheorem{definition}{定義}[section]
\newtheorem{thm}{定理}[section] 
\newtheorem{prf}{証明}[section]
\newtheorem{exmp}{例}[section]

\begin{document}
\maketitle{}
\section{導入}
\section{産業背景}
\subsection{スイッチOTCとは}
セルフメディケーションの推進や医療費の適正化のために、医療用医薬品として使われている成分が要望をもとに審査を経て、市販薬(以下、OTC(over-the-counter)とも呼ぶ)として販売可能となることがある。この市販薬はスイッチOTC\footnote{スイッチOTCはOTCの部分集合である。}と呼ばれる。\cite{jmsi2020}によれば、スイッチOTCとなるのは、「医療用医薬品としての使用実績があり、有効性・安全性が確立されている」ものである。承認されるのは、年に数成分程度である。
\subsection{セルフメディケーション税制}
セルフメディケーション税制とは、スイッチOTCを購入した際に、その購入費用について所得控除を受けられるものである\footnote{セルフメディケーション税制(特定の医薬品購入額の所得控除制度)について\url{https://www.mhlw.go.jp/stf/seisakunitsuite/bunya/0000124853.html}}。しかし、購入額が世帯合計で12000円以上でなくてはならず、医療費控除と同時に利用できないなど条件が厳しいため、あまり利用されていない。\cite{jmsi2021}によれば、セルフメディケーション税制利用者は、2019年分が推計で2万人程度、確定申告者ベースで0.10\%となっている。したがって、影響は小さいと考えられ、本稿ではセルフメディケーション税制の効果は考慮していない。2022年4月1日から、対象が一部非スイッチOTCにも拡大されたが、本研究で使用するのはそれ以前であるか、以降であってもセルフメディケーション税制対象かどうかの情報を用いないため、セルフメディケーション税制対象であることとスイッチOTCであることを同義として扱う。
\subsection{薬価基準(医療用医薬品の薬剤費)}
以降の薬価の制度情報は主に\cite{takahashi}に依拠する。薬価基準は保健医療に使用できる医薬品の品目表としての役割と、使用された薬剤の請求額を定めた価格表としての役割がある。主に医療用医薬品に対して定められていて\footnote{例外は存在する。}、小売価格を決定しているととらえられる。なお、本稿では消費者が直面する医療用医薬品の価格は、薬価だけではなく、調剤料、診察料などを含めた医療費の自己負担額だとしている。

\subsubsection{新医薬品の薬価}
新医薬品は、類似薬のあるものは類似薬効比較方式、類似薬のないものは原価計算方式で薬価が決定される。類似薬効比較方式では、類似薬に比して有用性がある場合は様々な加算がなされる。いずれの場合も外国平均価格調整という、外国価格との調整が行われる。類似薬効比較方式の場合は規格間の調整も行われる。
\subsubsection{既収載医薬品の薬価改定}
2018年までは2年に一度、2019年以降は毎年改定されている。改定は、市場実勢価格(卸売価格)と薬価の差を解消するために行われている。卸売価格は規制されていないため、病院や薬局は薬価との間の利ざやをとることができる。\cite{Iizuka2007}によると薬価改定は次のような数式で表される。
\begin{align*}
p^r_t=p^w_{t-1}+p^r_{t-1}\times R
\end{align*}
ここで\(p^r_t\)は\(t\)期の薬価、\(p^w_{t-1}\)が\(t-1\)期の卸売価格、\(R\)が調整幅で2\%である。
\subsubsection{後発医薬品の統一名収載}
後発医薬品には原則3つの価格帯があり、一番低いものは統一名収載される。つまり、個別の医薬品ごとには薬価が設定されず、かつ個別の医薬品名も薬価基準に収載されないということである。
\subsection{市販薬(OTC)の価格}
市販薬は企業が自由に価格を決定できる。全体として、スイッチOTCは非スイッチOTCより高い価格で販売される傾向にある\citep{iseikyoku2021}。

\subsection{薬効分類}
本稿では薬効分類というとき、日本の医療用医薬品の薬効分類をさす。OTCの薬効分類も存在しているが、医療用医薬品のものとの対応が不明確であり、本稿では使用していない。薬効分類は4桁まであり、階層的になっていて、1桁増えるごとに情報が増え、分類が細かくなる\footnote{例えば、KEGGではこのように可視化されている。\url{https://www.kegg.jp/brite/jp08301}}。本稿では、基本的に3桁、利用可能であれば4桁を使用している。
\section{データ} 
データは、薬事工業生産動態統計とNDBオープンデータを主に用いた。
\subsection{薬事工業生産動態統計}
薬事工業生産動態統計は、厚生労働省が実施している医薬品等の生産の実態を把握するための調査(薬事工業生産動態統計調査)によって作成されている。調査は毎月行われていて、月次と年次のデータがある。本研究では、月次データは2009年1月から2023年4月までの15年4か月分を、年次データは2008年から2021年までの14年分を使用した。様々な値がレポートされているが、医薬品の国内向け出荷額\footnote{国産と輸入品の区別をしていない。}と国内生産額\footnote{主成分の数において、国産より輸入のほうが多いものを含む。外国製造業者が最終製造工程を行っていないもの。}と年末在庫額を使用した。これらは薬効分類3桁の単位で報告されている。薬事工業生産動態統計の医療用医薬品(先発と後発の区別がない)とOTC(セルフメディケーション税制対象の区別あり)のデータを使用した。
\subsection{NDBオープンデータ}
NDBオープンデータ\footnote{【NDB】NDBオープンデータ\url{https://www.mhlw.go.jp/stf/seisakunitsuite/bunya/0000177182.html}}とは、厚生労働省のレセプト情報・特定健診等情報データベース(NDB、National Database)から個票データを集計し公表してあるものである。そのうちの薬剤に関するデータを用いた。内服、外用、注射に対して、それぞれ、外来(院内)、外来(院外)、入院のデータがあるが、このうち、スイッチOTCと関連すると考えられる内服か外用の外来のデータのみを用いた。2014年度分から始まり、2021年度分までの年次データが2023年12月現在公表されている。薬剤データに関しては、初回は各薬効分類(3桁)の上位30品目の処方量(単位は1日分)、薬価などが報告されていたが、2回目以降は上位100品目に変更されている。したがって、2015年度から2021年度のデータを用いた。 
\subsection{KEGG DRUG データベース}
KEGG DRUGデータベースは、「日本、米国、欧州の医薬品情報を化学構造と成分の観点から一元的に集約したデータベース」\footnote{KEGG DRUG データベース\url{https://www.genome.jp/kegg/drug/drug_ja.html}}である。ここから、スイッチOTCの成分とスイッチ承認年、薬効分類(4桁)を得た\footnote{ 日本のスイッチOTC薬\url{https://www.genome.jp/kegg/drug/jp08314.html}}。
\subsection{スイッチOTCの発売日のデータ}
KEGG DRUGデータベースのスイッチ承認年のデータは、年までの情報しかない。また、承認から発売までのラグが医薬品によって異なる。よって各成分のスイッチOTCとしての最初の発売日(年月日)のデータを取得した\footnote{日までわからない場合は15日とした。}。情報源は、薬事日報、PR TIMES、各社プレスリリース、日経テレコンなどである。
\subsection{薬価基準}
前述の薬価基準のデータも使用した。薬価基準には製造企業の情報が記載されていて、これを利用するためである。既収載品の薬価は基本的に改定年の4月1日に改定されるが、途中で追加される医薬品もある。したがって、改定直前のデータを取得し、改定前の薬価とした。使用した期間は2015年度から2021年度までである。NDBオープンデータとは、薬価基準収載医薬品コードと年で結合した。
\subsection{MEDISの医薬品HOTコードマスター}
薬価基準では統一名収載の医薬品の製造企業情報が得られない。しかし、NDBオープンデータでは統一名収載の医薬品であっても、個別の医薬品名でレポートされているものが多い。そこでNDBのマスタであり、統一名収載だがDB上は個別名である医薬品の製造企業の情報も含まれている、MEDISの医薬品HOTコードマスター\footnote{医薬品HOTコードマスター\url{https://www2.medis.or.jp/master/hcode/}}のデータをレセプト電算コードと年で結合した。
\subsection{後発医薬品の使用割合}
後発医薬品の使用割合(数量シェア)はデータの観測期間において急激に引き上げられている。データは厚生労働省の資料\footnote{後発医薬品の使用割合の目標と推移\url{https://www.mhlw.go.jp/content/000890777.pdf}}から取得した。ただし、『「使用割合」とは、「後発医薬品のある先発医薬品」及び「後発医薬品」を分母とした「後発医薬品」の使用割合』である。隔年でしか調査されていない時期は線形補間した。これは全体の値であり、薬効分類ごとの値ではない。薬効分類ごとには、NDBオープンデータで後発品の数量シェアを計算した\footnote{後発医薬品のない先発医薬品も含んでおり、政策目標に使われているものとは定義が異なる。}。また、これをテストデータと訓練データにわけ、NDBオープンデータの期間外における薬効分類ごとの後発品の数量シェアをランダムフォレストで予測して補完した変数を作成した。
\begin{align*}
\widehat {後発品の数量シェア}_{it}= \textrm{RandomForest}(後発品の数量シェア_t,\mathbbm{1}\{薬効分類_i\})
\end{align*}
\subsection{OTCの出荷額を数量にする}
薬事工業生産動態統計では出荷額がレポートされているが価格の情報はない。NDBオープンデータの単位である1日分の数量にそろえるために、\cite{igarashi2021}の1日分あたりのOTCの平均価格の情報\footnote{資料中に特に記載がないがこれはスイッチOTCの価格であると解釈した。}を用いた。1日分あたりのOTCの平均価格でOTCの出荷額を割り、1日分の数量に換算した。OTCの1パッケージあたりの平均価格の情報は\cite{iseikyoku2021}にあるが、そこから1パッケージあたりの数量が\(N\)日分か
\[N= \textrm{1パッケージあたりの平均価格}/\textrm{OTCの1日分の平均価格}\]
を計算した。同情報から、非スイッチOTCの価格はスイッチOTCの0.7倍とした。この計算はスイッチOTCと非スイッチOTCでそれぞれ行い、それぞれ約11日分と約13日分となった。これはある程度リーズナブルな値だと思われる。また、出荷額と販売量は異なるが、販売量の情報はなかったため、出荷額を販売量として扱った。
\subsection{データの処理}
異質性があると考えられるため、1度も成分がスイッチしていない薬効分類(3桁)のデータは除外した。0や欠損値も除外した。
\subsection{価格に関する仮定}
消費者が直面する価格は、3割負担の医療費、スイッチOTCの価格、非スイッチOTCの価格であるとした。ただし、すべて1日分である。これらの値はすべて取得することができなかったため、\cite{igarashi2021,igarashi2022}や\cite{iseikyoku2021}の価格比の情報をもとに個別の薬価や薬効分類の平均薬価の定数倍とした。 それは次のような関係式である。
\begin{align}
    \widehat {3割負担の医療費}_{jt}&=3.4\times 薬価_{jt}\\
    \widehat {\textrm{スイッチOTCの価格}}_t&=0.5 \times \widehat{\textrm{3割負担の医療費}}_t\\
    \widehat {\textrm{非スイッチOTCの価格}}_t&=0.7 \times \widehat {\textrm{スイッチOTCの価格}}_t
\end{align}

(1)式から、医療費に占める薬剤費の割合を約8\%と仮定していることになる。これは近年国民医療費に占める薬剤費の割合が20\%程度で推移していることと乖離しているが、スイッチOTCが存在するような薬効分類にはあまり高額な医薬品が存在しないと考えられるため、ある程度理にかなった値だと考えられる。

(2)式に関連して、\cite{narui2013}や\cite{narui2016}では支払意思額のアンケートをスポーツクラブや薬局で行っている。そこでは、医療用医薬品を購入するのにかかる負担額(診察代とお薬代)を2000円とした場合に、それぞれ平均的にその65\%や75\%程度の価格ならばスイッチOTCを購入したいという結果が得られている。企業が支払意思額までチャージできるとは限らないため、50\%という値はある程度リーズナブルだと考えられる。

(3)式に関しては、医療用と同成分、同量配合などのプレスリリースや広告が多く出されていることから、スイッチOTCにはある種のブランド価値があり、非スイッチOTCより高く売られているということが考えられる。
\section{モデル}
イベントスタディと離散選択モデルによる需要推定を行った。
\subsection{イベントスタディ}
スイッチOTC化の影響を見るためにイベントスタディを行った。トリートメントはスイッチ成分のOTCとしての最初の発売とした。計量モデルには固定効果モデルを使用した。データは、薬事工業生産動態統計とNDBオープンデータをそれぞれ別に使用した。
次のような特定化を使用した。
\begin{align*}
\log y_{it} =\sum_k \gamma_k D_{kit}+\alpha_i +\delta_t+\beta X_{it}+\epsilon_{it}
\end{align*}
ここで各変数は次のようなものである。
\begin{align*}
y_{it}&: 医療用医薬品または\textrm{OTC} の薬効分類ごとの出荷額または生産額(薬事工業生産動態統計)、\\
& \quad 先発医薬品または後発医薬品の薬効分類ごとの処方量( \textrm{NDB} オープンデータ)\\
D_{kit}&: 時間tにおける、薬効分類iでスイッチした成分が最初に \textrm{OTC} として発売されてから\\
& \quad k期間後(前)を表すダミー変数\\
\alpha_i&: 薬効分類の固定効果\\
\delta_t&: 時間の固定効果\\
X_{it}&: その他のコントロール変数\\
    &\quad 共通: \log(後発品の数量シェア_{it})\\
    &\quad 薬事工業生産動態統計の場合: \log(1期前の在庫額_i)\\
    & \quad \textrm{NDB} オープンデータの場合: \log(後発品の数量シェア_t) 
\end{align*}
ここで、\(i\)は薬効分類(薬事工業生産動態統計の場合は3桁、NDBオープンデータの場合は4桁)、\(t\)は時間である。年次と月次のデータがあるため\(t\)の単位は年か月である。\(k\)の単位はデータが年次の場合は年であり、月次の場合は年、クォーター、月の場合で推定した。\(D_{-1it}\)は除外して正規化している。
\subsection{需要推定}
\cite{Berry1994}や\cite{BLP}の離散選択モデルによる需要推定を行った。

\section{推定結果}

\section{考察}

\section{結論} 

\newpage
\bibliography{thesis} %hoge.bibから拡張子を外した名前
\bibliographystyle{jecon}
\end{document}