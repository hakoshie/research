\documentclass[a4paper,12pt,uplatex]{jsarticle}
\title{\vspace{-3cm}医療用医薬品の市販薬への転用\\(スイッチOTC化)の影響}
\author{橋之口浩平\thanks{一橋大学大学院経済学研究科修士課程2年 e-mail: em225025@g.hit-u.ac.jp}}
\date{}
\usepackage{amsmath}
\usepackage{amssymb}
\usepackage{amsthm}
\usepackage{ascmac}
\usepackage{bm}
\usepackage{comment}
\usepackage{color}
\usepackage{enumerate}
%\usepackage{eclbkbox}
\usepackage{empheq}
\usepackage{float}
\usepackage[dvipdfmx]{graphicx}
\usepackage[dvipdfmx]{hyperref}
\usepackage{mathrsfs}
\usepackage{natbib}
\usepackage{pxjahyper}
%\usepackage{physics}
%\usepackage{sgamex}
%\usepackage{listings,jvlisting}
%\usepackage{tikz}
%\renewcommand{\labelitemi}{$\blacktriangleright$}
\PassOptionsToPackage{hyphens}{url}
\def\UrlBreaks{\do\/\do-}
\bibpunct[:]{(}{)}{,}{a}{,}{,}
\let\Iff\Leftrightarrow
\let\trm\textrm
\let\tbf\textbf
\let\tc\textcolor{}{}
\let\mf\mathfrak
\let\msc\mathscr
\let\msr\mathscr
\let\ms\mathscr
\let\ov\overline
\let\To\Rightarrow
\let\l\left
\let\r\right
\theoremstyle{definition}
\newtheorem{definition}{定義}[section]
\newtheorem{thm}{定理}[section] 
\newtheorem{prf}{証明}[section]
\newtheorem{exmp}{例}[section]

\begin{document}
\maketitle{}
\section{導入}
\section{産業背景}
セルフメディケーションの推進や医療費の適正化のために、医療用医薬品として使われている成分が市販薬(以下、OTC(over-the-counter)とも呼ぶ)として販売可能となることがある。この市販薬はスイッチOTCと呼ばれる。
\subsection{薬価(医療用医薬品の薬剤費)}
薬価の制度情報は主に\cite{takahashi}による。薬価は医療用医薬品に対して定められていて、小売価格を決定している。なお、本稿では消費者が直面する医療用医薬品の価格は、薬価だけではなく、調剤料、診察料などを含めた医療費の自己負担額だとしている。

\subsubsection{新規収載医薬品の薬価}

\subsubsection{既収載医薬品の薬価改定}
2018年までは2年に一度、2019年以降は毎年改定されている。改定は、市場実勢価格と薬価の差を解消するために行われている。卸売価格は規制されていないため、病院や薬局は薬価との間の利ざやをとることができる。
\subsection{市販薬(OTC)の価格}
市販薬は企業が自由に価格を決定できる。全体として、スイッチOTCは非スイッチOTCより高い価格で販売される傾向にある。
\section{データ}

\section{モデル}
\section{推定結果}
\section{考察}

\section{結論}

\bibliography{thesis} %hoge.bibから拡張子を外した名前
\bibliographystyle{jecon}
\end{document}