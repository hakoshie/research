\documentclass[a4paper,12pt,uplatex]{jsarticle}
\title{\vspace{-3cm}医療用医薬品の市販薬への転用\\(スイッチOTC化)の影響}
\author{橋之口浩平\thanks{一橋大学大学院経済学研究科修士課程2年 e-mail: em225025@g.hit-u.ac.jp}}
\date{}
\usepackage{amsmath}
\usepackage{amssymb}
\usepackage{amsthm}
\usepackage{ascmac}
\usepackage{bm}
\usepackage{comment}
\usepackage{color}
\usepackage{enumerate}
%\usepackage{eclbkbox}
\usepackage{empheq}
\usepackage{float}
\usepackage[dvipdfmx]{graphicx}
\usepackage[dvipdfmx]{hyperref}
\usepackage{mathrsfs}
\usepackage{natbib}
\usepackage{pxjahyper}
%\usepackage{physics}
%\usepackage{sgamex}
%\usepackage{listings,jvlisting}
%\usepackage{tikz}
%\renewcommand{\labelitemi}{$\blacktriangleright$}
\PassOptionsToPackage{hyphens}{url}
\def\UrlBreaks{\do\/\do-}
\bibpunct[:]{(}{)}{,}{a}{,}{,}
\let\Iff\Leftrightarrow
\let\trm\textrm
\let\tbf\textbf
\let\tc\textcolor{}{}
\let\mf\mathfrak
\let\msc\mathscr
\let\msr\mathscr
\let\ms\mathscr
\let\ov\overline
\let\To\Rightarrow
\let\l\left
\let\r\right
\theoremstyle{definition}
\newtheorem{definition}{定義}[section]
\newtheorem{thm}{定理}[section] 
\newtheorem{prf}{証明}[section]
\newtheorem{exmp}{例}[section]

\begin{document}
\maketitle{}
\section{導入}
\section{産業背景}
\subsection{スイッチOTCとは}
セルフメディケーションの推進や医療費の適正化のために、医療用医薬品として使われている成分が市販薬(以下、OTC(over-the-counter)とも呼ぶ)として販売可能となることがある。この市販薬はスイッチOTCと呼ばれる。
\subsection{薬価基準(医療用医薬品の薬剤費)}
以降の薬価の制度情報は主に\cite{takahashi}に依拠する。薬価基準は保健医療に使用できる医薬品の品目表としての役割と、使用された薬剤の請求額を定めた価格表としての役割がある。主に医療用医薬品に対して定められていて\footnote{例外は存在する}、小売価格を決定しているととらえられる。なお、本稿では消費者が直面する医療用医薬品の価格は、薬価だけではなく、調剤料、診察料などを含めた医療費の自己負担額だとしている。

\subsubsection{新医薬品の薬価}
新医薬品は、類似薬のあるものは類似薬効比較方式、類似薬のないものは原価計算方式で薬価が決定される。類似薬効比較方式では、類似薬に比して有用性がある場合は様々な加算がなされる。いずれの場合も外国平均価格調整という、外国価格との調整が行われる。類似薬効比較方式の場合は規格間の調整も行われる。
\subsubsection{既収載医薬品の薬価改定}
2018年までは2年に一度、2019年以降は毎年改定されている。改定は、市場実勢価格(卸売価格)と薬価の差を解消するために行われている。卸売価格は規制されていないため、病院や薬局は薬価との間の利ざやをとることができる。\cite{Iizuka2007}によると薬価改定は次のような数式で表される。
\begin{align*}
p^r_t=p^w_{t-1}+p^r_{t-1}\times R
\end{align*}
ここで\(p^r_t\)は\(t\)期の薬価、\(p^w_{t-1}\)が\(t-1\)期の卸売価格、\(R\)が調整幅で2\%である。
\subsection{市販薬(OTC)の価格}
市販薬は企業が自由に価格を決定できる。全体として、スイッチOTCは非スイッチOTCより高い価格で販売される傾向にある\citep{iseikyoku2021}。
\subsection{薬効分類}
本稿では薬効分類というとき、日本の医療用医薬品の薬効分類をさす。OTCの薬効分類も存在しているが、医療用医薬品のものとの対応が不明確であり、本稿では使用していない。薬効分類は4桁まであり、階層的になっていて、1桁増えるごとに分類が細かくなる\footnote{例えば、KEGGではこのように可視化されている。\url{https://www.kegg.jp/brite/jp08301}}。本稿では、3桁か4桁を使用している。
\section{データ} 
データは、薬事工業生産動態統計とNDBオープンデータを主に用いた。
\subsection{薬事工業生産動態統計}
薬事工業生産動態統計は、厚生労働省が実施している医薬品等の生産の実態を把握するための調査(薬事工業生産動態統計調査)によって作成されている。調査は毎月行われていて、月次と年次のデータがある。本研究では、月次データは2009年1月から2023年4月までの15年4か月分を、年次データは2008年から2021年までの14年分を使用した。様々な値がレポートされているが、医薬品の国内向け出荷額\footnote{国産と輸入品の区別をしていない}と、国内生産額\footnote{主成分の数において、国産より輸入のほうが多いものを含む}(外国製造業者が最終製造工程を行っていないもの)の合計を使用した。これらは薬効分類3桁の単位で報告されている。
\subsection{NDBオープンデータ}
NDBオープンデータ\footnote{\url{https://www.mhlw.go.jp/stf/seisakunitsuite/bunya/0000177182.html}}とは、厚生労働省のレセプト情報・特定健診等情報データベース(NDB\footnote{National Databaseの略。})から個票データを集計し公表してあるものである。そのうちの薬剤に関するデータを用いた。内服、外用、注射に対して、それぞれ、外来(院内)、外来(院外)、入院のデータがあるが、このうち、スイッチOTCと関連すると考えられる内服か外用の外来のデータのみを用いた。2014年度分から始まり、2021年度分までの年次データが2023年12月現在公表されている。薬剤データに関しては、初回は各薬効分類(3桁)の上位30品目の処方量(単位は1日分)、薬価などが報告されていたが、2回目以降は上位100品目に変更されている。したがって、2015年度から2021年度のデータを用いた。
\subsection{KEGG DRUG データベース}


\section{モデル}
\section{推定結果}
\section{考察}

\section{結論}  

\newpage
\bibliography{thesis} %hoge.bibから拡張子を外した名前
\bibliographystyle{jecon}
\end{document}