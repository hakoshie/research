\documentclass[a4paper,11pt,uplatex]{jsarticle}
\title{\vspace{-3cm}医療用医薬品の市販薬への転用\\(スイッチOTC化)の影響}
\author{橋之口浩平\thanks{一橋大学大学院経済学研究科修士課程2年 E-mail: em225025@g.hit-u.ac.jp}}
\date{}
\usepackage{amsmath}
\usepackage{amssymb}
\usepackage{amsthm}
\usepackage{ascmac}
\usepackage{bm}
\usepackage{comment}
\usepackage{color}
\usepackage{enumerate}
%\usepackage{eclbkbox}
\usepackage{empheq}
\usepackage{float}
\usepackage[dvipdfmx]{graphicx}
\usepackage[dvipdfmx]{hyperref}
\usepackage{mathrsfs}
\usepackage{natbib}
\usepackage{pxjahyper}
\usepackage{bbm}
\usepackage{booktabs}
\usepackage{physics}
\usepackage{appendix}
\usepackage{siunitx}
%\usepackage{sgamex}
%\usepackage{listings,jvlisting}
%\usepackage{tikz}
%\renewcommand{\labelitemi}{$\blacktriangleright$}
\PassOptionsToPackage{hyphens}{url}
\def\UrlBreaks{\do\/\do-}
\bibpunct[:]{(}{)}{,}{a}{,}{,}
\let\Iff\Leftrightarrow
\let\trm\textrm
\let\tbf\textbf
\let\tc\textcolor{}{}
\let\mf\mathfrak
\let\msc\mathscr
\let\msr\mathscr
\let\ms\mathscr
\let\ov\overline
\let\To\Rightarrow
\let\l\left
\let\r\right
\theoremstyle{definition}
\newtheorem{definition}{定義}[section]
\newtheorem{thm}{定理}[section] 
\newtheorem{prf}{証明}[section]
\newtheorem{exmp}{例}[section]
\renewcommand*{\thefootnote}{\arabic{footnote}}
\begin{document}
\maketitle{}
\section{導入}
本研究では、日本における医療用医薬品の市販薬(以下、OTC\footnote{over-the-counterの略}とも呼ぶ)への転用(スイッチOTC化)の影響を分析する。OTCによる潜在的医療費削減効果は\cite{igarashi2021}で試算してあり、レセプトの情報から置き換え可能とされているものが3200億円となっている。しかし、ここでは医療用医薬品とOTCの代替関係が考慮されておらず、実際にそのような置き換えが可能なのかについては疑問が残る。そこで本研究では、スイッチOTC化の影響を定量化するために、固定効果モデルによるイベントスタディや\cite{Berry1994}や\cite{BLP}の消費者の離散選択モデルによる需要推定を行った。

関連する研究としては、以下のものがある。\cite{Iizuka2007}では日本の医療用医薬品市場を離散選択モデルで分析し、医師と患者の間のプリンシパルエージェント問題を指摘している。\cite{Stomberg2013}では、アメリカでのスイッチOTC化の影響を薬効レベルや成分レベルで中断時系列分析を用いて分析し、スイッチOTC化により医薬品の使用が増えることを示している。\cite{Mahecha2006}はスイッチOTC化のケーススタディを行い、成功には参入のタイミングやマーケティングが大切であるとしている。\cite{medicare1995}は健康保険のカバーする範囲による医療用医薬品とOTCの使用の変化を分析し、健康保険に入っている人ほど、医療用医薬品の使用が増え、OTCの使用が減ることを確かめている。\cite{medicare2003}は2腕意思決定モデルで第二世代抗ヒスタミン薬のOTC化の影響を分析し、医療費を減少させるだけでなく、事故なども減少させ社会的にもよいものだったとしている。
\section{産業背景}
\subsection{スイッチOTCとは}
セルフメディケーションの推進や医療費の適正化のために、医療用医薬品として使われている成分が要望をもとに審査を経て、市販薬として販売可能となることがある。このような市販薬はスイッチOTC\footnote{スイッチOTCはOTCの部分集合である。スイッチOTCでないOTCを非スイッチOTCと呼ぶことにする。}と呼ばれる。\cite{jmsi2020}によれば、スイッチOTCとなるのは、「医療用医薬品としての使用実績があり、有効性・安全性が確立されている」ものである。承認されるのは、年に数成分程度である。スイッチOTC医薬品有効成分リスト\footnote{スイッチOTC医薬品有効成分リスト\url{https://www.mhlw.go.jp/content/10800000/000895247.pdf}}には91の成分が掲載されている。承認は1983年から始まっている。
\subsection{セルフメディケーション税制}
セルフメディケーション税制とは、2017年1月から開始した、スイッチOTCを購入した際にその購入費用について所得控除を受けられるものである\footnote{セルフメディケーション税制(特定の医薬品購入額の所得控除制度)について\url{https://www.mhlw.go.jp/stf/seisakunitsuite/bunya/0000124853.html}}。しかし、購入額が世帯合計で12000円以上でなくてはならず、医療費控除と同時に利用できないなど条件が厳しいため、あまり利用されていない。\cite{jmsi2021}によれば、セルフメディケーション税制利用者は、2019年分が推計で2万人程度、確定申告者ベースで0.10\%となっている。したがって、影響は小さいと考えられ、本稿ではセルフメディケーション税制の効果は考慮していない。2022年4月1日から、対象が一部非スイッチOTCにも拡大されたが、本研究で使用するのはそれ以前であるか、以降であってもセルフメディケーション税制対象かどうかの情報を用いないため、セルフメディケーション税制対象であることとスイッチOTCであることを同義として扱う。
\subsection{薬価基準(医療用医薬品の薬剤費)}
以降の薬価の制度情報は主に\cite{takahashi}に依拠する。薬価基準には保健医療に使用できる医薬品の品目表としての役割と、使用された薬剤の請求額を定めた価格表としての役割がある。薬価基準は主に医療用医薬品に対して定められていて\footnote{例外は存在する。}、(薬剤費の)小売価格を決定しているととらえられる。なお、本稿では消費者が直面する医療用医薬品の価格は、薬価だけではなく、調剤料、診察料などを含めた医療費の自己負担額としている。

\subsubsection{新医薬品の薬価}
新医薬品は、類似薬のあるものは類似薬効比較方式、類似薬のないものは原価計算方式で薬価が決定される。類似薬効比較方式では、類似薬に比して有用性がある場合は様々な加算がなされる。いずれの場合も外国平均価格調整という、外国価格との調整が行われる。類似薬効比較方式の場合は規格間の調整も行われる。
\subsubsection{既収載医薬品の薬価改定}
2018年までは2年に一度、2019年以降は毎年改定されている。改定は、市場実勢価格(卸売価格)と薬価の差を解消するために行われている。卸売価格は規制されていないため、病院や薬局は薬価との間の利ざやをとることができる。\cite{Iizuka2007}によると薬価改定は次のような数式で表される。
\begin{align*}
p^r_t=p^w_{t-1}+p^r_{t-1}\times R
\end{align*}
ここで\(p^r_t\)は\(t\)期の薬価、\(p^w_{t-1}\)が\(t-1\)期の卸売価格、\(R\)が調整幅で2\%である。
\subsubsection{後発医薬品の統一名収載}
後発医薬品には原則3つの価格帯があり、一番低いものは統一名収載される。つまり、個別の医薬品ごとには薬価が設定されず、かつ個別の医薬品名も薬価基準に収載されないということである。
\subsection{市販薬(OTC)の価格}
市販薬は企業が自由に価格を決定できる。全体として、スイッチOTCは非スイッチOTCより高い価格で販売される傾向にある\citep{iseikyoku2021}。

\subsection{薬効分類}
本稿では薬効分類というとき、日本の医療用医薬品の薬効分類をさす。OTCの薬効分類も存在しているが、医療用医薬品のものとの対応が不明確であるため本稿では使用していない。薬効分類は4桁まであり階層的になっている。1桁増えるごとに情報が増え、分類が細かくなる\footnote{例えば、KEGGではこのように可視化されている。\url{https://www.kegg.jp/brite/jp08301}}。本稿では、基本的に3桁、利用可能であれば4桁を使用している。
\section{データ} 
データは、薬事工業生産動態統計とNDBオープンデータを主に用いた。そのほか、KEGG DRUGデータベースやスイッチOTCの発売日のデータ、薬価基準、MEDISの医薬品HOTコードマスター、後発医薬品の使用割合などを使用した。以下でそれぞれのデータについて説明する。
\subsection{薬事工業生産動態統計}
薬事工業生産動態統計は、厚生労働省が実施している医薬品等の生産の実態を把握するための調査(薬事工業生産動態統計調査)によって作成されている。調査は毎月行われていて、月次と年次のデータがある。本研究では、月次データは2009年1月から2023年4月までの15年4か月分を、年次データは2008年から2021年までの14年分を使用した。様々な値がレポートされているが、医薬品の国内向け出荷額\footnote{国産と輸入品の区別をしていない。}と国内生産額\footnote{主成分の数において、国産より輸入のほうが多いものを含む。外国製造業者が最終製造工程を行っていないもの。}と12月末または月末の在庫額を使用した。これらは薬効分類3桁の単位で報告されている。特に医療用医薬品(先発と後発の区別がない)とOTC(年次の場合はセルフメディケーション税制対象の区分がある)のデータを使用した。
\subsection{NDBオープンデータ}
NDBオープンデータ\footnote{【NDB】NDBオープンデータ\url{https://www.mhlw.go.jp/stf/seisakunitsuite/bunya/0000177182.html}}とは、厚生労働省のレセプト情報・特定健診等情報データベース(NDB、National Database)から個票データを集計し公表してあるものである。そのうちの薬剤に関するデータを用いた。内服、外用、注射に対して、それぞれ、外来(院内)、外来(院外)、入院のデータがあるが、このうち、スイッチOTCと関連すると考えられる内服か外用の外来のデータのみを用いた。公表は2014年度分から始まり、2021年度分までの年次データが2023年12月現在ある。薬剤データに関しては、初回は各薬効分類(3桁)の上位30品目のデータが報告されていたが、2回目以降は上位100品目に変更されている。したがって、2015年度から2021年度のデータを用いた。 データの内容は、処方量(単位は1日分)や薬価、いくつかの医薬品コードなどである。
\subsection{KEGG DRUG データベース}
KEGG DRUGデータベースは、「日本、米国、欧州の医薬品情報を化学構造と成分の観点から一元的に集約したデータベース」\footnote{KEGG DRUG データベース\url{https://www.genome.jp/kegg/drug/drug_ja.html}}である。ここから、スイッチOTCの成分とスイッチ承認年、薬効分類(4桁)を得た\footnote{ 日本のスイッチOTC薬\url{https://www.genome.jp/kegg/drug/jp08314.html}}。
\subsection{スイッチOTCの発売日のデータ}
KEGG DRUGデータベースのスイッチ承認年のデータは、年までしか日付の情報がない。また、承認から発売までのラグが医薬品によって異なる。よって各成分のスイッチOTCとしての最初の発売日(年月日)のデータを取得した\footnote{日までわからない場合は15日とした。}。情報源は、薬事日報、PR TIMES、各社プレスリリース、日経テレコンなどである。
\subsection{薬価基準}
薬価基準のデータも使用した。薬価基準には製造企業の情報が記載されていて、これを利用するためである。既収載品の薬価は基本的に改定年の4月1日に改定されるが、途中で追加される医薬品もある。したがって、改定直前のデータを取得し、改定前の薬価基準として利用した。使用した期間は2015年度から2021年度までである。薬価基準には、同一剤形・規格の後発医薬品がある先発医薬品(以下、長期収載品)や局方品の情報もあり、これも利用した。局方品は、日本薬局方に収載されているものである。日本薬局方に収載されるのは「保健医療上重要な医薬品、すなわち有効性及び安全性に優れ、医療上の必要性が高く、国内外で広く使用されているもの」である\footnote{局方品(日本薬局方収載医薬品)について\url{https://www.jga.gr.jp/jgapedia/column/_19361.html}}。NDBオープンデータとは、薬価基準収載医薬品コードと年で結合した。 
\subsection{MEDISの医薬品HOTコードマスター}
薬価基準では統一名収載の医薬品の製造企業情報が得られない。しかし、NDBオープンデータでは統一名収載の医薬品であっても、個別の医薬品名でレポートされているものが多い。そこでNDBのマスタであり、統一名収載だがDB上は個別名である医薬品の、製造企業の情報も含まれている、MEDISの医薬品HOTコードマスター\footnote{医薬品HOTコードマスター\url{https://www2.medis.or.jp/master/hcode/}}のデータをレセプト電算コードと年で結合した。HOTコードは乱立している医薬品コードを対応づけるために開発されたコードである\footnote{\url{http://helics.umin.ac.jp/files/HS001/HS001_rep20180525_MEDIS.pdf}}。
\subsection{後発医薬品の使用割合}
後発医薬品の使用割合(数量シェア)はデータの観測期間において急激に上昇している。データは厚生労働省の資料\footnote{後発医薬品の使用割合の目標と推移\url{https://www.mhlw.go.jp/content/000890777.pdf}}から取得した。ただし、『「使用割合」とは、「後発医薬品のある先発医薬品」及び「後発医薬品」を分母とした「後発医薬品」の使用割合』である。隔年でしか調査されていない時期は線形補間した。これは全体の値であり、薬効分類ごとの値ではない。薬効分類ごとには、NDBオープンデータで後発品の数量シェアを計算した\footnote{後発医薬品のない先発医薬品も含んでおり、政策目標に使われているものとは定義が異なる。}。また、これをテストデータと訓練データにわけ、NDBオープンデータの期間外における薬効分類ごとの後発品の数量シェアをランダムフォレストで予測して補完した変数を作成した。これは次のようなものである
\begin{align*}
\widehat {後発品の数量シェア}_{it}= \textrm{RandomForest}(後発品の数量シェア_t,\mathbbm{1}\{薬効分類i\},t).
\end{align*}
\subsection{OTCの出荷額を数量にする}
薬事工業生産動態統計では出荷額がレポートされているが価格の情報はない。NDBオープンデータの単位である1日分の数量にそろえるために、\cite{igarashi2021}の1日分あたりのOTCの平均価格の情報\footnote{資料中に特に記載がないがこれはスイッチOTCの価格であると解釈した。}を用いた。1日分あたりのOTCの平均価格でOTCの出荷額を割り、1日分の数量に換算した。1パッケージあたりのOTCの平均価格の情報は\cite{iseikyoku2021}にあるが、そこから1パッケージあたりの数量が\(N\)日分か
\[N= \textrm{1パッケージあたりの平均価格}/\textrm{OTCの1日分の平均価格}\]
を計算した。なお同情報から、非スイッチOTCの価格はスイッチOTCの0.7倍とした。この計算はスイッチOTCと非スイッチOTCでそれぞれ行い、それぞれ約11日分と約13日分となった。これはある程度理にかなった値だと思われる。また、出荷額と販売量は異なるが、販売量の情報はなかったため、出荷額を販売量として扱った。
\subsection{データの処理}\label{sec:data_process}
異質性があると考えられるため、1度も成分がスイッチしていない薬効分類(3桁)のデータは除外した。0や欠損値も除外した。また、対数をとった後発品の数量シェアを使用するため、NDBオープンデータの上位100品目に後発品がない(薬効分類,年)のデータも除外した。
\subsection{価格に関する仮定}
消費者が直面する価格は、3割負担の医療費、スイッチOTCの価格、非スイッチOTCの価格であるとした。ただし、すべて1日分である。これらの値はすべて取得することができなかったため、\cite{igarashi2021,igarashi2022}や\cite{iseikyoku2021}の価格比の情報をもとに個別の薬価や薬効分類の平均薬価の定数倍とした。 それは次のような関係式である。
\begin{align}
    \widehat {3割負担の医療費}_{jt}&=3.4\times 薬価_{jt}\\
    \widehat {\textrm{スイッチOTCの価格}}_t&=0.5 \times \widehat{\textrm{3割負担の医療費}}_t\\
    \widehat {\textrm{非スイッチOTCの価格}}_t&=0.7 \times \widehat {\textrm{スイッチOTCの価格}}_t
\end{align}
ここでの\(j,t\)は表\ref{tab:demand_}のものである。

(1)式から、医療費に占める薬剤費の割合を約8\%と仮定していることになる。これは近年国民医療費に占める薬剤費の割合が20\%程度で推移していることと乖離しているが、スイッチOTCが存在するような薬効分類にはあまり高額な医薬品が存在しないと考えられるため、ある程度理にかなった値だと考えられる。

(2)式に関連して、\cite{narui2013}や\cite{narui2016}では支払意思額のアンケートをスポーツクラブや薬局で行っている。そこでは、医療用医薬品を購入するのにかかる負担額(診察代とお薬代)を2000円とした場合に、それぞれ平均的にその65\%や75\%程度の価格ならばスイッチOTCを購入したいという結果が得られている。企業が支払意思額までチャージできるとは限らないため、50\%という値はある程度理にかなっていると考えられる。

(3)式に関しては、医療用と同成分、同量配合などのプレスリリースや広告が出されていることから、スイッチOTCにはある種のブランド価値があり、非スイッチOTCより高く売られていると考えられる。

\subsection{データの概観}
記述統計は以下のようになった。ただし\ref{sec:data_process}の処理を行ったあとのデータについてのものである。

表\ref{desc:yakuji_month}は月次の薬事工業生産動態統計の記述統計である。医療用医薬品の平均出荷額が約140億円、OTCの平均出荷額が約13億円となっている。OTCの市場規模はおよそ医療用医薬品の1/10程度だとわかる。
\sisetup{round-mode = places, round-precision = 3,drop-zero-decimal}
\begin{table}[H]
    \centering     \caption{薬事工業生産動態統計 月次}\label{desc:yakuji_month}
    \begin{tabular}{lSS}
        \hline
        & 医療用医薬品& OTC  \\
        \hline
        観測数& \num{2752} & \num{2752} \\
        mean & \num{1.378616e4} & \num{1.311866e3} \\
        std & \num{1.450122e4} & \num{2.333219e3} \\
        min & \num{2.9835e1} & \num{1.2e-2} \\
        25\% & \num{1.805011e3} & \num{2.77145e1} \\
        50\% & \num{1.19982e4} & \num{2.485425e2} \\
        75\% & \num{1.981822e4} & \num{1.183112e3} \\
        max & \num{1.121442e5} & \num{2.966552e4} \\
        \hline
        \multicolumn{3}{l}{\small (単位:百万円)}
    \end{tabular}
\end{table}
表\ref{desc:yakuji_year}は年次の薬事工業生産動態統計の記述統計である。医療用医薬品の平均出荷額が約1500億円、OTCの平均出荷額が約150億円、スイッチOTCの平均出荷額が約70億円となっている。スイッチOTCの市場規模はOTCの1/2程度だとわかる。セルフメディケーション税制は2017年からであり、薬事工業生産動態統計でその対象が報告されているのは2015年からである。したがって観測数が他より少なくなっている。
\begin{table}[H]
    \centering     \caption{薬事工業生産動態統計 年次}\label{desc:yakuji_year}
    \begin{tabular}{lccc}
        \hline
        & 医療用医薬品& OTC & スイッチOTC \\
        \hline
        観測数& 213 & 213 & 112 \\
        mean & \num{1.552942e5} & \num{1.568697e4}& \num{7.019668e3} \\
        std & \num{1.484851e5} & \num{2.595221e4}& \num{1.495904e4} \\
        min & \num{8.74186e2} & \num{1.3818e1}& \num{3.2196e1} \\
        25\% & \num{2.533214e4} & \num{4.25903e2}& \num{1.514862e2} \\
        50\% & \num{1.530709e5} & \num{3.270722e3}& \num{5.0918e2} \\
        75\% & \num{2.411454e5} & \num{1.668112e4}& \num{4.52989e3} \\
        max & \num{8.654794e5} & \num{1.561291e5}& \num{1.034687e5} \\
        \hline
        \multicolumn{4}{l}{\small (単位:百万円)}
    \end{tabular}
\end{table}
表\ref{desc:ndb_len4}は、NDBオープンデータを薬効分類4桁、年、後発品ダミーで集計したものの記述統計である。対象となる薬効分類は32種類ある。平均処方量は約32億日分であり、1日分当たりの平均薬価の平均は約143円となっている。平均薬価の中央値は約26円と低くなっている。
\begin{table}[H]
    \centering
    \caption{NDBオープンデータ 薬効分類4桁、年、後発品ダミーで集計したもの}\label{desc:ndb_len4}
    \begin{tabular}{lSS}
    \hline
     & {処方量(100万日分)} & {平均薬価(円/1日分)} \\
    \hline
        観測数& \num{436} & 436.000000 \\
        mean & \num{3.174177e+03} & \num{143.317480 }\\
        std & \num{6.870684e+03}  & \num{392.741619 }\\
        min & \num{3.605813e-1}  & \num{5.400000 }\\
        25\% & \num{6.806452e+01} & \num{10.835614 }\\
        50\% & \num{2.786511e+02} & \num{25.557643 }\\
        75\% & \num{2.421159e+03} & \num{97.697571 }\\
        max & \num{3.766192e+4}  & \num{2899.505261 }\\
        \hline
      \end{tabular}
    \end{table}
    表\ref{desc:ndb}はNDBオープンデータの記述統計である。平均処方量は約1300万日分であり、1日分当たりの平均薬価は約357円となっている。薬価の中央値は約31円と低い。
    \begin{table}[H]
        \centering
        \caption{NDBオープンデータ}\label{desc:ndb}
        \begin{tabular}{lS[table-format=1.6e+08]S[table-format=5.6]}
        \hline
         & {処方量(100万日分)} & {薬価(円/1日分)} \\
        \hline
        観測数& \num{24673.000000} & \num{24673.000000} \\
        mean & \num{1.333124e+01} & \num{357.346671} \\
        std & \num{4.526902e+01} & \num{2079.828181} \\
        min & \num{7.012987e-4} & \num{0.660000} \\
        25\% & \num{3.518760e-1} & \num{12.200000} \\
        50\% & \num{2.266922e+0} & \num{31.400000} \\
        75\% & \num{9.503580e+0} & \num{110.400000} \\
        max & \num{9.541906e+02} & \num{80171.300000} \\
        \hline
    \end{tabular}
    \end{table}

\section{モデル}
イベントスタディと離散選択モデルによる需要推定を行った。
\subsection{イベントスタディ}
まず、スイッチOTC化の影響を見るためにイベントスタディを行った。処置はスイッチ成分のOTCとしての最初の発売とした。計量モデルには二元配置固定効果(Two way fixed effects, TWFE)モデルを使用した。データは、薬事工業生産動態統計とNDBオープンデータをそれぞれ別に使用した。
次のような特定化を使用した。
\begin{align*}
\log y_{it} =\sum_k \gamma_k D_{kit}+\alpha_i +\delta_t+\beta X_{it}+\epsilon_{it}
\end{align*}
ここで各変数は次のようなものである。
\begin{table}[H] \caption{変数の説明}
    \begin{tabular}{ll}
       \toprule 
       \(i\)&薬効分類(薬事工業生産動態統計は3桁、NDBオープンデータは4桁)\\
       \(t\)&時間。単位は年または月。(ただし、後発品の数量シェアの\(t\)は常に年)\\
       \(k\)&スイッチからの経過時間。単位は年、四半期、月。\\
    $y_{it}$ & 医療用医薬品またはOTC の薬効分類ごとの出荷額または生産額(薬事工業生産動態統計)、\\
    & 先発医薬品または後発医薬品の薬効分類ごとの処方量(NDBオープンデータ)\\
    $D_{kit}$ & 時間\(t\)における、薬効分類\(i\)でスイッチした成分が最初に OTC として発売されてから\\
    & \(k\)期間後(前)を表すダミー変数\\
    $\alpha_i$ & 薬効分類の固定効果\\
    $\delta_t$ & 時間の固定効果\\
    $X_{it}$ & その他のコントロール変数\\
    & 共通: $\log(後発品の数量シェア_{it})$\\
    & 薬事工業生産動態統計の場合: $\log(1期前の在庫額_i)$\\
    & \textrm{NDB} オープンデータの場合: $\log(後発品の数量シェア_t)$ \\
    \bottomrule
    \end{tabular}
\end{table}
\(D_{-1it}\)は除外して正規化している。月次データの場合は、経過時間の単位を年、月、四半期とした場合で推定した。
\subsection{需要推定}
\cite{Berry1994}や\cite{BLP}の差別化財の離散選択モデルによる需要推定を行った。価格弾力性の制約を緩めるためにNested Logitモデルを使用した。また、推定には\cite{ConlonGortmaker}によるPyBLPを使用した。記法も基本的にPyBLPのドキュメント\footnote{\url{https://pyblp.readthedocs.io/en/stable/notation.html}}に従っている。
モデルは次のようなものである。消費者は効用を最大化する医薬品を1つ選ぶ問題に直面している。
消費者\(i\)の市場\(t\)において医薬品\(j\)を購入することの間接効用は次のように表される。
\begin{align*}
U_{ijt}&=\alpha p_{jt} + x_{jt} \beta + \xi_{jt} + \bar\epsilon_{ih(j)t}+(1-\rho)\bar \epsilon_{ijt}, \quad \bar\epsilon_{ih(j)t}+(1-\rho) \bar \epsilon_{ijt} \sim \textrm{タイプI極値分布}\\
&=V_{jt}+\epsilon_{ijt}
\end{align*}
ここで\(p_{jt}\)は価格、\(x_{jt}\)は観測される製品特性、\(\xi_{jt}\)は観測されない製品特性である。消費者の嗜好を表す\(\epsilon_{ijt}=\bar\epsilon_{ih(j)t}+(1-\rho)\bar \epsilon_{ijt}\)はi.i.d.のタイプIの極値分布にしたがうとする。
平均効用は次のように表される。
\begin{align*}
\delta_{jt}=\log s_{jt}-\log s_{0t}-\rho \log s_{j | h(j)t}=\alpha p_{jt} + x_{jt} \beta +\xi_t+\Delta \xi_{jt}.
\end{align*}
ここで\(s_{0t}\)はアウトサイドグッズのシェア、\(h(j)\)は製品\(j\)の属するグループ、\(s_{j | h(j)t}\)はグループ内シェアであり、観測できない製品特性は固定効果と固定効果で説明できない部分にわけられる\(\xi_{jt}=\xi_t+\Delta \xi_{jt}\).
GMMの最小化問題は次のように表される。
\begin{align*}
\min_\theta q(\theta)=\bar g(\theta)'W \bar g(\theta).
\end{align*}
ここで
\begin{align*}
\bar g(\theta)=\frac{1}{N} \sum _{j,t} Z'_{jt} \Delta\xi_{jt}.
\end{align*}
重み行列については、まず操作変数の分散共分散行列の逆行列を用いて推定し、その後残差の分散共分散行列の逆行列に更新している。
以上における変数は次のように定義される。

\begin{table}[H]\centering \caption{変数の説明}
    \small
    \begin{tabular}{llll}
        \toprule \label{tab:demand_}
        \(j\) & 医薬品&$t$ &薬効分類(3桁)と年で分けられた市場 \\
        $p_{jt}$ &価格 & $q_{jt}$ &医薬品の数量(単位は1日分)\\ 
        $s_{jt}$ &シェア、$q_{jt}/M$ & $M$ &マーケットサイズ、100日分/人$\times$ 1.2億人\\
        $s_{j|h(j)t}$ & グループ内シェア & $h(j)$ &製品\(j\)の属するグループ\\
        $\xi_{jt}$ & 観測できない製品特性& $\xi_t$ &企業、薬効分類(3桁)、年の固定効果 \\
        $\Delta \xi_{jt}$ & 構造エラー &$Z_{jt}$ &市場\(t\)の製品\(j\)に関する操作変数 \\
        $x_{jt}$ & 後発品、内服薬、院内処方、局方品、& &  \\
         & 長期収載品、OTCのダミー & &\\
         \bottomrule
    \end{tabular}
\end{table}
ネストの構造は、先発医薬品か後発医薬品かOTCかとした。

操作変数にはBLP操作変数を使用した。使用例として、\cite{Iizuka2007}がある。各市場の自社の他の製品と他社の製品の製品特性の和を横に並べたものであり、市場の混雑度を表す。外生性に関しては、医薬品の研究開発には時間がかかるため、製品特性を簡単には変えられないという正当化がなされている。OTCに関しては、市場レベルの観測なので操作変数に使用しなかった。また、製品の数も加えた(和をとる製品特性として1を追加した)。

\section{推定結果} 
\subsection{イベントスタディ}
\subsubsection{薬事工業生産動態統計}
スイッチOTCの販売開始の、医療用医薬品とOTCの出荷額への影響のイベントスタディの結果は図\ref{otc_mon_y}、\ref{rx_mon_y}のようになった。これは月次データを用い、経過時間は年でカウントしたものである。経過時間の単位を月や四半期としたもの、年次データを用いたものは\ref{event_graph}にある。
また推定値や標準誤差などの詳細な結果は\ref{event_parameter}にある。 
\begin{figure}[H]
    \centering
    \begin{minipage}{0.45\textwidth}
        \caption{OTCの出荷額への影響(月次、経過年数)}\label{otc_mon_y}
        \centering
        \includegraphics[scale=0.3]{../estimate/event/plots/otc_log_mon_Y.png}
    \end{minipage}\hfill
    \begin{minipage}{0.45\textwidth}
        \caption{医療用医薬品の出荷額への影響(月次、経過年数)}\label{rx_mon_y}
        \centering
        \includegraphics[scale=0.3]{../estimate/event/plots/rx_log_mon_Y.png}
    \end{minipage}
\end{figure}
両方ともプレトレンドは見受けられない。OTCの出荷額は、販売開始後、最初の数年はおよそ35\%\((=\exp(0.3)-1)\)程度増えていて、その後はおよそ65\%\((=\exp(0.5)-1)\)程度増えている。医療用医薬品の出荷額については、販売開始前後でほとんど変化がなく、推定値もほとんど0の近くにある。
\subsubsection{NDBオープンデータ}
スイッチOTCの販売開始の、後発医薬品と先発医薬品の処方量への影響のイベントスタディの結果は図\ref{ndb_brand}、\ref{ndb_generic}のようになった。推定値や標準誤差などの詳細な結果は\ref{event_parameter}にある。
\begin{figure}[H]
    \centering
    \begin{minipage}{0.45\textwidth}
        \caption{先発医薬品の処方量への影響}\label{ndb_brand}
        \centering
        \includegraphics[scale=0.3]{../estimate/event/plots/ndb_brand.png}
    \end{minipage}\hfill
    \begin{minipage}{0.45\textwidth}
        \caption{後発医薬品の処方量への影響}\label{ndb_generic}
        \centering
        \includegraphics[scale=0.3]{../estimate/event/plots/ndb_generic.png}
    \end{minipage}
\end{figure}
両方とも、ゆるやかに右上がりのプレトレンドがあるように見える。先発医薬品の処方量については、推定値の符号が正の期間と負の期間があるが、おおよそ0の付近にある。後発医薬品の処方量については、販売開始後の推定値はほとんど0の近くにある。
\subsection{需要推定}
\subsubsection{間接効用}
間接効用のパラメータの推定値は表\ref{indirect_utility}のようになった。
\begin{table}[H]
    \centering
    \footnotesize
    \caption{推定結果}\label{indirect_utility}
    \begin{tabular}{l c}
        \toprule
        \multicolumn{2}{l}{Rho Estimate} \\
        \midrule
        Estimate & $0.478$\\
        % \midrule
        Robust SE & $(0.0494)$ \\
        \bottomrule
    \end{tabular}

    \vspace{1em}
    \footnotesize
    \begin{tabular}{lccccccc}
        \toprule
        \multicolumn{8}{l}{Beta Estimates} \\
        \midrule
        & 価格 & 院内処方 & 内服薬 & 後発品 & OTC & 長期収載品 & 局方品 \\
        \midrule
        Estimate & $-0.0696$ & $0.255$ & $1.23$ & $-0.585 $ & $-1.84 $ & $-0.140 $ & $0.130 $ \\
        Robust SE & $(0.0228 )$ & $(0.0258 )$ & $(0.126 )$ & $(0.0477)$ & $(0.680)$ & $(0.0499 )$ & $(0.0179 )$ \\
        \bottomrule
    \end{tabular}
\end{table}
ここで価格の単位は千円である。価格の係数の絶対値は、ほかのものに比べて相対的に小さい。院内処方の係数が正となっているのは薬局によるチェックがないことによる、プリンシパルエージェント問題が発生している可能性がある。内服薬の係数が大きくなっているが、外用薬が代替の選択肢となっていることは多くないと思われ、これは単にシェアが大きいためだと考えられる。後発品、OTC、長期収載品のダミーの係数が負であることから先発品が好まれていることが分かる。局方品は質が高いものだととらえられるため、係数が正となっていることは整合的である。
\subsubsection{価格弾力性}
代替関係を見るために価格弾力性を計算した。
価格弾力性は次のように定義されたメソッドで計算してある\footnote{価格弾力性の計算\url{https://pyblp.readthedocs.io/en/stable/_api/pyblp.ProblemResults.compute_elasticities.html}}。
\begin{align*}
\varepsilon_{jk}=\frac{p_{kt}}{s_{jt}} \pdv{s_{jt}}{p_{kt}}.
\end{align*}
シェアを使用しているが、\(s_{jt}=q_{jt}/M\)で\(M\)は定数なので、数量を使用している場合と同じである。

自己価格弾力性は表\ref{own_price_elasticity}のようになった。
\begin{table}[H]
    \centering 
    \caption{自己価格弾力性}\label{own_price_elasticity}
    \begin{tabular}{lccc}
        \hline
        & \text{医療用医薬品} & \text{スイッチOTC} & \text{非スイッチOTC} \\
        \hline
        Mean & $-0.162$ & $-0.0199$ & $-0.0232$ \\
        Standard Deviation & $0.945$ & $0.0448$ & $0.0510$ \\
        \hline
    \end{tabular}
\end{table}
医療用医薬品の自己価格弾力性については、\cite{Contoyannis2005}や\cite{Yeung2018}と同様の値となっている。

交差価格弾力性は表\ref{cross_price_elasticity}のようになった。
\begin{table}[H]
    \centering
    \caption{交差価格弾力性}\label{cross_price_elasticity}
    \begin{tabular}{lccc}
        \hline
        & \text{医療用医薬品} & \text{スイッチOTC} & \text{非スイッチOTC} \\
        \hline
        Mean & $2.33 \times 10^{-4}$ & $1.46 \times 10^{-5}$ & $1.59 \times 10^{-5}$ \\
        Standard Deviation & $2.52 \times 10^{-3}$ & $6.63 \times 10^{-5}$ & $6.31\times 10^{-5}$ \\
        \hline
    \end{tabular}
\end{table}
医療用医薬品の平均の絶対値や標準偏差は、そのほかに比べて大きくなっている。
\subsubsection{消費者余剰の変化} 
消費者余剰は次のように定義されたメソッドで計算してある\footnote{消費者余剰の計算\url{https://pyblp.readthedocs.io/en/latest/_api/pyblp.ProblemResults.compute_consumer_surpluses.html}}。
\begin{align*}
\textrm{CS}_{it}=\log \left.\left(1+\sum_{h \in H}\exp V_{iht}\right)\middle/\left(- \pdv{V_{i1t}}{p_{1t}}\right)\right. .
\end{align*} 
所得効果がないとしているため、基準とする医薬品はランダムに選ばれている。

仮想的にスイッチOTC、非スイッチOTC、OTCがなかった場合と現状の消費者余剰を計算した。その差をとったものが、スイッチOTC、非スイッチOTC、OTCへの支払い意思額のような値である。計算された消費者余剰の値は表\ref{consumer_surplus}のようになった。
\begin{table}[H]\centering \caption{消費者余剰}\label{consumer_surplus}
    \begin{tabular}{ccccc}
        \hline
        &現状 & スイッチOTCなし & 非スイッチOTCなし & OTCなし \\
        \hline
        \(\textrm{CS}_i\)(平均)&77995 & 76754 & 75215 & 73931\\
        \hline
    \end{tabular}
\end{table}
これは、年について平均をとった、年間の1人あたりの額(単位は円)である。厚生の改善は1人当たりOTC全体が4000円程度、スイッチOTCが1200円程度、非スイッチOTCが2800円程度となった。
\section{考察}
イベントスタディの結果から、スイッチOTCは医療用医薬品の需要をほとんど減少させない一方、OTCの市場を拡大させていることが示唆される。また、需要推定の価格弾力性の結果からは、医療用医薬品とOTCの代替関係は、医療用医薬品間の代替関係よりは弱いことが示唆される。以上から、OTCへのスイッチは消費者の医薬品へのアクセスを向上させ、OTCの市場を拡大させるが、医療用医薬品の代替はあまり期待できないと考えられる。同様のことは、\cite{Stomberg2013}でも指摘されている。

本研究で用いたTWFEモデルでは、処置のタイミングがずれている場合に負のウェイトが生じることがあり、処置効果がグループや時間に対して異質的であるときに問題になるということが\cite{ROTH2023}などで指摘されている。これがNDBオープンデータを用いたイベントスタディでのプレトレンドにつながっている可能性がある。また、コントロールグループが不明確であるため、TWFE推定量がしている比較を解釈することが難しいことも指摘されている。
そのほか、プレトレンドのテストの検出力が低いことも指摘されている。それらを解消するための新たな手法や検定が開発されているが、本研究ではそれらを用いなかった。
\section{結論}
本研究では、スイッチOTC化の影響を定量化するために、TWFEモデルを用いたイベントスタディや、\cite{Berry1994}や\cite{BLP}のモデルによる需要推定を行った。まず、イベントスタディに関しては、OTCの出荷額を30-65\%程度上昇させる効果が見られたが、医療用医薬品の出荷額や処方量にはあまり影響がないことがわかった。
次に、需要推定の価格弾力性からは、医療用医薬品とOTCの代替関係は、医療用医薬品間の代替関係よりは弱いことが示唆された。また、スイッチOTCによる厚生の改善は年間1人当たり1200円程度、非スイッチOTCは2800円程度と計算された。

以上の結果は、スイッチOTCは消費者の医薬品へのアクセスを向上させ、OTCの市場を拡大させるが、医療用医薬品を代替することはあまり期待できないことを示唆する。その背景にある理由としては、事前には自分が何の病気であるかわからないことや、医師や薬剤師に相談することに価値を感じる人が多いこと、代替するほどの価格差がないことなどが考えられる。

% 本研究は、価格に関する情報が不足している、OTCの観測が市場レベルであるなどデータが不完全であり、分析のためにアドホックな仮定を置いていることに注意が必要である。また、スイッチOTC化に伴う市販薬の乱用の増加なども考慮していない。
\bibliography{thesis} %hoge.bibから拡張子を外した名前
\bibliographystyle{jecon}
\begin{appendices}
    \def\thesection{付録\Alph{section}}
    \section{}\label{event_graph}
    \begin{figure}[H]
        \centering
        \begin{minipage}{0.45\textwidth}
            \caption{OTCの出荷額への影響(月次、経過月数)}
            \centering
            \includegraphics[scale=0.3]{../estimate/event/plots/otc_log_mon.png}
            \end{minipage}\hfill
        \begin{minipage}{0.45\textwidth}
            \caption{OTCの出荷額への影響(月次、経過四半期数)}
            \centering
            \includegraphics[scale=0.3]{../estimate/event/plots/otc_log_mon_Q.png}
        \end{minipage}
    \end{figure}
    \begin{figure}[H]
        \centering
        \begin{minipage}{0.45\textwidth}
            \caption{医療用医薬品の出荷額への影響(月次、経過月数)}
            \centering
            \includegraphics[scale=0.3]{../estimate/event/plots/rx_log_mon.png}
            \end{minipage}\hfill
        \begin{minipage}{0.45\textwidth}
            \caption{医療用医薬品の出荷額への影響(月次、経過四半期数)}
            \centering
            \includegraphics[scale=0.3]{../estimate/event/plots/rx_log_mon_Q.png}
        \end{minipage}
    \end{figure}
    \begin{figure}[H]
        \centering
        \begin{minipage}{0.45\textwidth}
            \caption{OTCの出荷額への影響(年次)}
            \centering
            \includegraphics[scale=0.3]{../estimate/event/plots/otc_log.png}
        \end{minipage}\hfill
        \begin{minipage}{0.45\textwidth}
            \caption{医療用医薬品の出荷額への影響(年次)}
            \centering
            \includegraphics[scale=0.3]{../estimate/event/plots/rx_log.png}
        \end{minipage}
    \end{figure}
    \section{推定結果の詳細}\label{event_parameter}
    \begin{table}[ht]
        \small
        \centering
        \caption{図\ref{otc_mon_y} OTCの出荷額への影響(月次、経過年数)の詳細}
        \begin{tabular}{lcccccc}
        \hline
        \textbf{Variable} & \textbf{Estimate} & \textbf{Std. Err.} & \textbf{T-stat} & \textbf{P-value} & \textbf{Lower CI} & \textbf{Upper CI} \\
        \hline
        D\_m10 & 0.0902 & 0.1482 & 0.6087 & 0.5427 & -0.2004 & 0.3808 \\
        D\_m9 & 0.1902 & 0.2212 & 0.8596 & 0.3901 & -0.2436 & 0.6240 \\
        D\_m8 & 0.1454 & 0.2296 & 0.6333 & 0.5266 & -0.3048 & 0.5956 \\
        D\_m7 & -0.2944 & 0.2462 & -1.1957 & 0.2319 & -0.7773 & 0.1884 \\
        D\_m6 & -0.0933 & 0.1231 & -0.7580 & 0.4485 & -0.3348 & 0.1481 \\
        D\_m5 & 0.1590 & 0.1272 & 1.2503 & 0.2113 & -0.0904 & 0.4084 \\
        D\_m4 & 0.1043 & 0.1715 & 0.6084 & 0.5430 & -0.2320 & 0.4406 \\
        D\_m3 & 0.2328 & 0.1833 & 1.2700 & 0.2042 & -0.1267 & 0.5923 \\
        D\_m2 & 0.2358 & 0.1521 & 1.5503 & 0.1212 & -0.0625 & 0.5341 \\
        D\_0 & 0.2853 & 0.0932 & 3.0617 & 0.0022 & 0.1026 & 0.4681 \\
        D\_1 & 0.3048 & 0.1149 & 2.6535 & 0.0080 & 0.0796 & 0.5300 \\
        D\_2 & 0.3867 & 0.1312 & 2.9480 & 0.0032 & 0.1295 & 0.6438 \\
        D\_3 & 0.3885 & 0.1492 & 2.6039 & 0.0093 & 0.0959 & 0.6811 \\
        D\_4 & 0.5874 & 0.1841 & 3.1916 & 0.0014 & 0.2265 & 0.9483 \\
        D\_5 & 0.5566 & 0.1586 & 3.5092 & 0.0005 & 0.2456 & 0.8676 \\
        D\_6 & 0.5855 & 0.1335 & 4.3856 & 0.0000 & 0.3237 & 0.8473 \\
        D\_7 & 0.5456 & 0.1156 & 4.7202 & 0.0000 & 0.3189 & 0.7722 \\
        D\_8 & 0.4748 & 0.1259 & 3.7718 & 0.0002 & 0.2279 & 0.7216 \\
        D\_9 & 0.4578 & 0.1581 & 2.8955 & 0.0038 & 0.1478 & 0.7678 \\
        D\_10 & 0.5347 & 0.1862 & 2.8723 & 0.0041 & 0.1697 & 0.8997 \\
        D\_11 & 0.4980 & 0.1632 & 3.0504 & 0.0023 & 0.1779 & 0.8181 \\
        D\_12 & 0.4664 & 0.1349 & 3.4573 & 0.0006 & 0.2019 & 0.7309 \\
        D\_13 & 0.5814 & 0.1355 & 4.2903 & 0.0000 & 0.3157 & 0.8472 \\
        D\_14 & 0.4826 & 0.1714 & 2.8156 & 0.0049 & 0.1465 & 0.8186 \\
        D\_15 & 0.4014 & 0.1415 & 2.8376 & 0.0046 & 0.1240 & 0.6788 \\
        log\_stock\_otc\_lag & 0.2697 & 0.0767 & 3.5158 & 0.0004 & 0.1193 & 0.4201 \\
        log(generic\_share\_it) & -0.2031 & 0.1765 & -1.1509 & 0.2499 & -0.5492 & 0.1429 \\
        \hline
        \end{tabular}
        \end{table}
        
    \begin{table}[ht]
        \centering \small
        \caption{図\ref{rx_mon_y} 医療用医薬品の出荷額への影響(月次、経過年数)の詳細}
        \begin{tabular}{lcccccc}
        \hline
        \textbf{Variable} & \textbf{Estimate} & \textbf{Std. Err.} & \textbf{T-stat} & \textbf{P-value} & \textbf{Lower CI} & \textbf{Upper CI} \\
        \hline
        D\_m10 & -0.1060 & 0.1278 & -0.8295 & 0.4069 & -0.3567 & 0.1446 \\
        D\_m9 & -0.0481 & 0.0806 & -0.5964 & 0.5510 & -0.2061 & 0.1100 \\
        D\_m8 & -0.0026 & 0.0780 & -0.0336 & 0.9732 & -0.1556 & 0.1503 \\
        D\_m7 & 0.0187 & 0.0688 & 0.2710 & 0.7864 & -0.1163 & 0.1536 \\
        D\_m6 & 0.1395 & 0.0790 & 1.7656 & 0.0776 & -0.0154 & 0.2944 \\
        D\_m5 & 0.0214 & 0.0570 & 0.3749 & 0.7078 & -0.0904 & 0.1332 \\
        D\_m4 & 0.0235 & 0.0593 & 0.3964 & 0.6919 & -0.0928 & 0.1398 \\
        D\_m3 & -0.0281 & 0.0508 & -0.5532 & 0.5802 & -0.1278 & 0.0716 \\
        D\_m2 & 0.0379 & 0.0564 & 0.6711 & 0.5022 & -0.0728 & 0.1485 \\
        D\_0 & 0.0499 & 0.0499 & 0.9999 & 0.3175 & -0.0480 & 0.1477 \\
        D\_1 & 0.0093 & 0.0494 & 0.1875 & 0.8513 & -0.0876 & 0.1061 \\
        D\_2 & 0.0456 & 0.0498 & 0.9166 & 0.3595 & -0.0520 & 0.1432 \\
        D\_3 & 0.0709 & 0.0676 & 1.0483 & 0.2946 & -0.0617 & 0.2035 \\
        D\_4 & 0.0130 & 0.0543 & 0.2387 & 0.8113 & -0.0935 & 0.1194 \\
        D\_5 & 0.0214 & 0.0540 & 0.3969 & 0.6914 & -0.0844 & 0.1272 \\
        D\_6 & 0.0563 & 0.0598 & 0.9405 & 0.3470 & -0.0610 & 0.1736 \\
        D\_7 & 0.0674 & 0.0487 & 1.3827 & 0.1669 & -0.0282 & 0.1629 \\
        D\_8 & 0.0604 & 0.0509 & 1.1870 & 0.2353 & -0.0394 & 0.1603 \\
        D\_9 & 0.0498 & 0.0372 & 1.3371 & 0.1813 & -0.0232 & 0.1228 \\
        D\_10 & 0.0427 & 0.0413 & 1.0336 & 0.3014 & -0.0383 & 0.1238 \\
        D\_11 & -0.0061 & 0.0299 & -0.2051 & 0.8375 & -0.0648 & 0.0526 \\
        D\_12 & -0.0036 & 0.0392 & -0.0925 & 0.9263 & -0.0805 & 0.0733 \\
        D\_13 & 0.0644 & 0.0413 & 1.5605 & 0.1188 & -0.0165 & 0.1453 \\
        D\_14 & 0.0051 & 0.0274 & 0.1871 & 0.8516 & -0.0486 & 0.0589 \\
        D\_15 & 0.0629 & 0.0470 & 1.3385 & 0.1809 & -0.0293 & 0.1551 \\
        log\_stock\_rx\_lag & 0.3108 & 0.0595 & 5.2279 & 0.0000 & 0.1942 & 0.4274 \\
        log(generic\_share\_it) & -0.0710 & 0.0757 & -0.9385 & 0.3481 & -0.2194 & 0.0774 \\
        \hline
        \end{tabular}
    \end{table}
     
    \begin{table}[ht]
    \small
    \centering  
    \caption{図\ref{ndb_brand} 先発医薬品の処方量への影響の詳細}
        \begin{tabular}{lcccccc}
            \hline
            \textbf{Variable} & \textbf{Estimate} & \textbf{Std. Err.} & \textbf{T-stat} & \textbf{P-value} & \textbf{Lower CI} & \textbf{Upper CI} \\
            \hline
            D\_m7 & -0.2452 & 0.1149 & -2.1344 & 0.0344 & -0.4721 & -0.0183 \\
            D\_m6 & -0.3699 & 0.1631 & -2.2678 & 0.0247 & -0.6920 & -0.0477 \\
            D\_m5 & -0.2869 & 0.0852 & -3.3673 & 0.0010 & -0.4551 & -0.1186 \\
            D\_m4 & -0.1739 & 0.0521 & -3.3394 & 0.0010 & -0.2768 & -0.0710 \\
            D\_m3 & -0.1641 & 0.0504 & -3.2560 & 0.0014 & -0.2636 & -0.0645 \\
            D\_m2 & -0.0590 & 0.0417 & -1.4175 & 0.1583 & -0.1413 &  0.0232 \\
            D\_0 & -0.0074 & 0.0548 & -0.1346 & 0.8931 & -0.1157 &  0.1009 \\
            D\_1 &  0.0245 & 0.0486 &  0.5044 & 0.6147 & -0.0715 &  0.1205 \\
            D\_2 &  0.0916 & 0.0507 &  1.8047 & 0.0730 & -0.0086 &  0.1918 \\
            D\_3 &  0.0767 & 0.0527 &  1.4565 & 0.1472 & -0.0273 &  0.1808 \\
            D\_4 & -0.1384 & 0.0613 & -2.2585 & 0.0253 & -0.2595 & -0.0174 \\
            D\_5 & -0.0761 & 0.0611 & -1.2444 & 0.2152 & -0.1968 &  0.0447 \\
            D\_6 & -0.0561 & 0.0618 & -0.9067 & 0.3660 & -0.1782 &  0.0661 \\
            D\_7 & -0.0924 & 0.0628 & -1.4706 & 0.1434 & -0.2165 &  0.0317 \\
            D\_8 & -0.0401 & 0.0615 & -0.6515 & 0.5157 & -0.1616 &  0.0814 \\
            D\_9 &  0.0078 & 0.0619 &  0.1258 & 0.9001 & -0.1145 &  0.1300 \\
            D\_10 & 0.0230 & 0.0474 & 0.4866 & 0.6272 & -0.0705 & 0.1166 \\
            log(generic\_share\_t) & 4.7156 & 0.0263 & 179.32 & 0.0000 & 4.6637 & 4.7676 \\
            log(generic\_share\_it) & -0.1465 & 0.0656 & -2.2330 & 0.0270 & -0.2760 & -0.0169 \\
            \hline
        \end{tabular}
        \label{tab:panelols_summary}
    \end{table}
    
        
    \begin{table}[ht]
    \small 
    \centering
    \caption{図\ref{ndb_generic} 後発医薬品の処方量への影響の詳細}
    \begin{tabular}{lcccccc}
        \hline
        \textbf{Variable} & \textbf{Estimate} & \textbf{Std. Err.} & \textbf{T-stat} & \textbf{P-value} & \textbf{Lower CI} & \textbf{Upper CI} \\
        \hline
        D\_m7 & -0.2129 & 0.0666 & -3.1988 & 0.0017 & -0.3444 & -0.0814 \\
        D\_m6 & -0.2736 & 0.0933 & -2.9333 & 0.0039 & -0.4578 & -0.0894 \\
        D\_m5 & -0.1923 & 0.0552 & -3.4847 & 0.0006 & -0.3013 & -0.0833 \\
        D\_m4 & -0.1209 & 0.0289 & -4.1816 & 0.0000 & -0.1779 & -0.0638 \\
        D\_m3 & -0.1068 & 0.0335 & -3.1898 & 0.0017 & -0.1730 & -0.0407 \\
        D\_m2 & -0.0715 & 0.0242 & -2.9610 & 0.0035 & -0.1193 & -0.0238 \\
        D\_0 & 0.0087 & 0.0440 & 0.1968 & 0.8442 & -0.0782 & 0.0955 \\
        D\_1 & -0.0081 & 0.0272 & -0.2966 & 0.7672 & -0.0618 & 0.0457 \\
        D\_2 & -0.0123 & 0.0405 & -0.3045 & 0.7612 & -0.0923 & 0.0676 \\
        D\_3 & 0.0046 & 0.0416 & 0.1099 & 0.9126 & -0.0775 & 0.0867 \\
        D\_4 & -0.0416 & 0.0441 & -0.9418 & 0.3477 & -0.1287 & 0.0456 \\
        D\_5 & -0.0411 & 0.0427 & -0.9619 & 0.3376 & -0.1254 & 0.0433 \\
        D\_6 & 0.0079 & 0.0438 & 0.1799 & 0.8574 & -0.0787 & 0.0945 \\
        D\_7 & 0.0006 & 0.0443 & 0.0137 & 0.9891 & -0.0868 & 0.0880 \\
        D\_8 & 0.0033 & 0.0446 & 0.0734 & 0.9416 & -0.0848 & 0.0914 \\
        D\_9 & 0.0176 & 0.0394 & 0.4456 & 0.6565 & -0.0603 & 0.0954 \\
        D\_10 & 0.0291 & 0.0286 & 1.0177 & 0.3104 & -0.0274 & 0.0856 \\
        log(generic\_share\_t) & 4.8703 & 0.0114 & 425.41 & 0.0000 & 4.8477 & 4.8929 \\
        log(generic\_share\_it) & 0.9797 & 0.0202 & 48.561 & 0.0000 & 0.9399 & 1.0196 \\
        \hline
        \end{tabular}
        \end{table}
\end{appendices}
\end{document}